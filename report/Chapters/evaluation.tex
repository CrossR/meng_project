\chapter{Performance Evaluation}%
\label{eval}

This chapter, the performance of both of the networks outlined in
Chapter~\ref{implem} will be evaluated, such that their performance at playing
the game StarCraft II can be measured. The exact methodology for these tests is
described in Chapter~\ref{eval_method}.

Each section will be split into three, first evaluating the Deep Q network,
second the Convolutional network, and then finally comparing the two. The
results will mainly focus around the score achieved, which is the in-game is the
final score for the mini-game at hand, and for the Simple64 map this will
compare how the bot fairs across 5 games against various bots. For the
mini-games the overall comparison will allow comparisons with known baselines,
such that we can show how the agents stack up. For the Simple64 game where this
is not possible, it will instead focus on how the agent plays.

%TODO: I just made up the number here for how we will test the Simple64 map...
%so we should actually pick a number and make sure to update the above bit.

After this has been done for both the mini-games and the Simple64 map, the
results will compared overall to see what the advantages and disadvantages of
each network is, as well as comparing other parts of the networks such as
training time, and robustness.

Finally, before the conclusion, some of the additional parts of the trained
networks which are not directly linked to the performance of the network, but
do give insight into how the networks works will be shown. For example the
filters for the Convolutional network will be given here, as well as some
example outputs.

\section{Mini-games}

\subsection{Deep Q Network}

\subsection{Convolutional Network}

\subsection{Network comparison}

\section{The Simple64 Map}

\subsection{Deep Q Network}

\subsection{Convolutional Network}

\subsection{Network comparison}

\section{Overall comparison}
% Overall comparison over both Accuracy and Generalisation and the networks as a
% whole, probably also including mention of training time.

\section{Network Information}
% Other additional parts of the network.

\section{Conclusion}
% Broad conclusions of the networks based on the their performances.
