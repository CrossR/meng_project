\chapter{Conclusion}%
\label{conclusion}

\section{Expansions of System}

As mentioned throughout the project, there is many things that could be
investigated, and other already included features that could be expanded.

To start, as spoken about briefly before, we could expand the Deep Q network,
such that it has a much richer action space. This would have two advantages.
Firstly, the agent should be able to learn more complex games and be more
accurate, as currently the action space is the main thing holding the agent
back. This should make it score higher and as such, move closer to a more
intelligent and optimal solution. Secondly, this would have the advantage of
allowing the agent to behave more like a human. That is, despite the agents
compound actions giving it an advantage in the learning process, it is also
unrealistic as compared to a human player, who does not have the option of one
action that then performs many actions. Hopefully with a large enough action
space the agent would be able to learn these actions by itself.

Next, the convolutional network could be expanded. Firstly, this should include
the inclusion of non-spatial information such as resource and unit counts, such
that they can help influence decisions. This would bring it in-line with the
`FullyConv' specification that DeepMind speaks about in their paper. Similarly,
after this, the network could be expanded further to include a Long-term short
term memory component, which is the next network DeepMind specified. These two
modifications would allow the network to have more information, as well as
retain information on past decisions to help make the current decision.

As mentioned during the research, a network that uses a teacher/student mechanic
could be used, to see if it helps it learn the more complex mini-games, or also
if it helps time to train on the simpler games. This could be applied to both
networks, to provide similar comparisons as transfer and curriculum learning were
used for here.

Finally, something the networks would potentially also benefit from is using a
later version of the SC2LE, which has pixel level classifications, rather than
the broad feature maps that are available in the most recent version. This could
help the convolutional method most, but would also raise many issues. The
current network architecture performs very well using a very small number of
convolutional layers since the images it uses are feature maps with very
distinct features. If this was instead swapped to pixel level data, then the
network could potentially be able to become more accurate as it would have the
smaller models rather than the blobs produced by the feature maps, but the
network would also need to be expanded suitably to be able to process the
images rather than feature maps.

Any of the features mentioned here could provide interesting further work for
a project, with the chance of also increasing the performance of the networks
outlined here.

\section{Deliverables}
%TODO: Link this back to the deliverables we outlined at the start of the paper,
%and how it stacks up, did we achieve it etc?

\section{Conclusion}
%TODO: Final conclusion of the paper, what did we learn, how did it go, what can
%be done to improve it and so on.

